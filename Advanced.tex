%
% Advanced.tex
% Advanced Techniques
%
% LulzBot® Mini User Manual
%
% Copyright (C) 2014 Aleph Objects, Inc.
%
% This document is licensed under the Creative Commons Attribution 4.0
% International Public License (CC BY-SA 4.0) by Aleph Objects, Inc.
%
%

\section{\texttt{Intro}}
\index{advanced techniques}
\glossary{3D Printer}{Also referred to as additive manufacturing, is the process of fabricating objects from 3D model data, through the deposition of a material in accumulative layers.}
After you become familiar with printing using the default settings, a few advanced techniques may help in getting better and more consistent prints from the Mini 3D printer. Some of these instructions are items and materials not included with the Mini. With any of these additional items or materials, follow safety and usage guidelines as instructed by the manufacturer.


\section{\texttt{Changing nozzles}}
\index{nozzle}
\glossary{Nozzle}{The metal tip at the bottom of the hot end. It has a small hole where the plastic filament is extruded out.}
\index{high resolution}
Your hot end is equipped with a 0.50mm nozzle. This nozzle diameter will print faster than a 0.35mm nozzle. Due to the specific torque- 30 ft/lb- required to tighten the nozzle when removed, removing the nozzle is not recommended. Failure to properly tighten the nozzle to the specific recommended torque may lead to leaks or damage if over-tightened. \textcolor{red}{Nozzle related issues will not be covered under warranty after nozzle changes}. We recommend having another tool head on hand with the alternative nozzle diameter for quick, easy, and simple changes.

\glossary{Layer height}{The thickness of each individual deposited layer of the three-dimensional model when cut with a slicing program.}
\glossary{Heater block}{Machined from aluminum, the heater block generates heat with a heater resistor and uses a thermistor to measure the temperature.}
\glossary{Thermistor}{A special type of resistor that changes resistance based on temperature. It is used to measure temperature on the nozzle and the heated bed.}
\glossary{Heater resistor}{A special type of resistor that is used to apply heat in a small area.}

\section{\texttt{Bed Adhesion}}
\index{warping}
\index{bottle}
\index{Ninjaflex}
\index{Nylon}
\index{brim}
\index{PEI}
\index{IPA}
\index{Isopropyl alcohol}
You may find that during printing, printed parts lift off the print surface on the corners. If you are seeing this turning on \texttt{Brim} in Cura or Slic3r will help increase the surface area of the print, improving part adhesion. If the corners of the part still lift, clean the PEI surface with IPA/ Isopropyl Alcohol and sand the surface with fine grit (2000-3000) sandpaper. While almost all 3D printing filaments will adhere well to the PEI print surface, Nylon print adhesion can be improved with the application of a PVA glue solution/gluestick. Flexible filaments such as Ninjaflex\textsuperscript{\miniscule{\textregistered}} can adhere too well and can be released easier from the PEI print surface with an application of a PVA glue solution/glue stick to the heated bed prior to starting the printing process.




%\section{ABS/Acetone Glue}
%\label{sec:ABS/Acetone Glue}
%\index{acetone}
\glossary{Acetone}{A colorless, volatile, flammable liquid ketone, (CH3)2CO, used as a solvent for ABS.}
Acetone is not included, required, or recommended for use with the Mini 3D printer.}

%\index{ABS}
%\textcolor{red}{Acetone can cause skin irritation when prolonged skin contact occurs. We recommend using acetone safe gloves when applying the ABS/acetone glue. Use the ABS/acetone glue in a well-ventilated space. Leave the mixture bottle closed except when applying a small amount to the wiping towel. Acetone liquid and vapors are highly flammable. Keep acetone away from open flames and high temperature sources, including the 3D printer. Read the warnings label on your purchased acetone packaging for additional warnings.}

\index{PEI}
\glossary{PEI}{Polyetherimide.}
The Mini 3D printer has a print surface of PEI. This printing surface can be used without any additional bed treatment. Keep the PEI sheet clean by wiping it with Isopropyl Alcohol and by lightly sanding the PEI surface with 2000-2500 grit sand paper.

\definecolor{green2}{rgb}{0.00,0.375,0.0}

\section{Firmware Flashing}
\label{ssec:num2}
When updates to your printers firmware become available, they will be automatically loaded into our latest version of Cura that can be found at lulzbot.com/cura. Once downloaded and installed, be sure the correct machine is selected by running through the add new machine process \texttt{Machine > Add New Machine} and then follow the directions. Once the machine is selected, power on your printer and plug it into your computer. Next go to \texttt{Machine > Install Default Firmware.} Once the progress bar is filled, your firmware flash will be complete.

\section{Z Offset}
\label{ssec:Z Offset}
\index{Z offset}
\index{first layer height}
Your LulzBot Mini 3D printer has the ability to change the first layer height (Z offset) directly through Cura. In the lower right hand corner of the Control Window in cura, enter the following commands. (The green text explains the command, do not enter it.)
\begin{itemize}
\item \texttt{M851}                    \textcolor{green2}{; This will report your current Z offset}
\item \texttt{M851 ZXXX}                    \textcolor{green2}{; This will update your Z offset to XXX}
\item \texttt{M500}                    \textcolor{green2}{; Save new offset}
\end{itemize}

To decrease the Z-offset between your nozzle tip and the print surface (decreasing the gap,) subtract from your current offset when changing XXX. To increase the Z-offset between the nozzle tip and the print surface (increasing the gap,) add to your current offset when changing XXX. When updating your offset, be sure to make small changes between prints. \textcolor{red}{We recommend making changes using small, 0.1mm increments at a time.} 

\subsection{\texttt{ABS/Acetone Glue}}
\label{sec:ABS/Acetone Glue}
\index{acetone}
\index{ABS}
\index{bed adhesion}

\glossary{Acetone}{A colorless, volatile, flammable liquid ketone, (CH3)2CO, used as a solvent for ABS.}
Acetone is not included or required with the LulzBot TAZ 3D printer. \textcolor{red}{Acetone and ABS solution is NOT recommended anymore, as the PEI print surface can be damaged by Acetone.}


\section{\texttt{Using 1.75mm filament}}
\index{1.75mm filament}

Your LulzBot Mini 3D printer is set up to use 3mm plastic filament by default and may be capable of printing 1.75mm filament with no hardware modification. While many of our advanced users are able to do so, your results may vary.  More information can be found in our User Forums at: \texttt{https://forum.lulzbot.com/viewtopic.php?f=16\&t=1923} 
